\documentclass[a4paper, 12pt]{article}

\usepackage[top=2cm, bottom=3cm, right=2cm, left=3cm]{geometry}
\usepackage[utf8]{inputenc}
\usepackage{amsmath, amsfonts, amssymb}
\usepackage{float}
\usepackage{graphicx}
\usepackage[portuguese]{babel}
\usepackage{hyperref}

\title{INART - Atividade 17}
\author{Luís Otávio Lopes Amorim - SP3034178}

\newcommand{\solucao}{
\begin{center}
\textbf{SOLUÇÃO}
\end{center}}

\begin{document}
\maketitle
\begin{enumerate}
\item Interprete o meme.

 A graça do meme é que o Capitão América fez a análise da acurácia da rede utilizando os mesmos dados que foram utilizados para o treino. Isso é um grande problema, já que uma grande acurácia medida no dataset de treino não indica uma alta performance, como o Capitão afirma no primeiro quadro, indica apenas que a rede pode ter decorado bem esse conjunto utilizado para o treino.
\end{enumerate}
\end{document}
