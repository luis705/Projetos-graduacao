	\item Seja o operador linear $T:\mathbb{R}^3 \rightarrow \mathbb{R}^3$
	definido pela matriz
	$\left[
	\begin{array}{ccc}
		1 & 0 & 1 \\
		2 & -1 & 1 \\
		0 & 0 & -1
	\end{array}
	\right]$
	
	\begin{enumerate}
		%	LETRA A		%
		\item Mostrar que $T$ é isomorfismo.
		\\ \\
		\solucao
		
		$$
		\left[
		\begin{array}{ccc}
			1 & 0 & 1 \\
			2 & -1 & 1 \\
			0 & 0 & -1
		\end{array}
		\right]
		\left[
		\begin{array}{c}
			x \\
			y \\
			z 
		\end{array}
		\right]
		=
		\left[
		\begin{array}{c}
			x + z\\
			2x - y +z \\
			-z 
		\end{array}
		\right]
		$$
	\\ \\
	 $$\left\{
	 \begin{array}{lll}
	 	1x + 0y + 1z = 0 \\
	 	2x - 1y + 1z = 0  \\
	 	0x + 0y - 1z = 0
	 \end{array}
	 \right.$$
	\\ \\
	\paragraph{} Como a única solução desse sistema é a solução homogênea, ou seja
	$x = y = z = 0$, temos que $Ker(T) = \{0\}$. Por isso, pelo
	teorema do núcleo o da imagem, sabemos que $dim V = dim Im$, ou
	seja, $T$ é isomorfismo.
	
	%	LETRA B		%
	\item Determinar a lei que define o operador $T^{-1}$
	\\ \\
	\solucao
	
	$$\left(
	\begin{array}{ccc|ccc}
		1  & 0  & 1  & 1  & 0  &  0 \\
		2  & -1 & 1  & 0  & 1  &  0 \\
		0  & 0  & -1 & 0  & 0  &  1
	\end{array}
	\right)
	\begin{array}{c}
		L_1 = L_1 + L_3 \\
		L_2 = L_2 + L_3 \\
		L_3 = -L_3	
	\end{array}
	\left(
	\begin{array}{ccc|ccc}
		1  & 0  & 0  & 1  & 0  &  1 \\
		2  & -1 & 0  & 0  & 1  &  1 \\
		0  & 0  & 1  & 0  & 0  & -1
	\end{array}
	\right)
	\begin{array}{c}
		\\
		L_2 = 2L_1 - L_2 \\
	\end{array}
	$$
	
	$$\left(
	\begin{array}{ccc|ccc}
		1  & 0  & 0  & 1  & 0  &  1 \\
		0  & 1  & 0  & 2  & -1 &  1 \\
		0  & 0  & 1  & 0  & 0  & -1
	\end{array}
	\right)
	$$
	\\ \\
	Portanto, temos:
	
	$$
	\left[T\right]^{-1} = 
	\left(
	\begin{array}{ccc}
		1  & 0  &  1 \\
		2  & -1 &  1 \\
		0  & 0  & -1
	\end{array}
	\right)
	\, \, \,
	T^{-1}(x, y, z) = (x + 2y, -y, x + y - z)
	$$
	\end{enumerate}
