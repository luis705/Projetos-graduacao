\item Calcular os autovalores e autovetores das
seguintes matrizes

\begin{enumerate}
	%	LETRA A 	%
	\item 
	$A = 
	\begin{bmatrix}
		 1 & 3 \\
		-1 & 5
	\end{bmatrix}		
	$
	\pagebreak
	\solucao
	$$det
	\begin{bmatrix}
		 1 - \lambda & 3 \\
		-1           & 5 - \lambda
	\end{bmatrix}
	=
	(1 - \lambda)(5-\lambda) + 3	
	=
	\lambda^2 - 6\lambda+8 = 0	
	$$
	$$
	\Delta = 6^2 - 4\cdot 1 \cdot 8 = 4
	\lambda = \frac{6 \pm \sqrt{4}}{2}
	$$
	\begin{center}
	$\lambda_1 = 2$
	$\lambda_2 = 4$	
	\end{center}
	Agora, encontrando os autovetores:
	
	$$
	\begin{pmatrix}
		-1 & 3 \\
		-1 & 3
	\end{pmatrix}
	x = 3y
	\, \, \, 
	v_1 = (3, \, 1)
	$$
	
	$$
	\begin{pmatrix}
		-3 & 3 \\
		-1 & 1
	\end{pmatrix}
	x = y
	\, \, \, 
	v_2 = (1, \, 1)
	$$
	
	Portanto, temos:
	
	$$
	\begin{array}{cc}
		\lambda_1 = 2 & \lambda_2 = 4 \\
		v_1 = (3, \, 1) & v_2 = (1, \, 1)
	\end{array}		
	$$
	
	%	LETRA B 	%
	\item
	$A = 
	\begin{bmatrix}
		3 &  3 & -1 \\
		0 & -1 &  0 \\
		8 &  6 & -5
	\end{bmatrix}		
	$
	\\ \\
	\solucao
	$$det
	\begin{pmatrix}
		3 - \lambda &  3           & -1           \\
		0           & -1 - \lambda &  0           \\
		8           &  6           & -5 - \lambda 
	\end{pmatrix}
	= (3 - \lambda)
	\begin{vmatrix}
		-1 - \lambda & 0           \\
		 6           & -5 - \lambda
	\end{vmatrix}
	-3
	\begin{vmatrix}
		0 &  0           \\
		8 & -5 - \lambda
	\end{vmatrix}
	-2
	\begin{vmatrix}
		0 & -1 - \lambda \\
		8 &  6          
	\end{vmatrix}
	$$
	\\
	\begin{center}
		$
		(3 - \lambda)(1+\lambda)(5+\lambda) - 16(1+\lambda) =
		(1+\lambda)\left[(3-\lambda)(5+\lambda) - 16\right]=0		
		$
		
		$
		(1+\lambda)(-\lambda^2-2\lambda-1) =
		-(1+\lambda)(\lambda^2+2\lambda+1) = -(1+\lambda)^3 = 0 \Rightarrow \lambda = -1$
	\end{center}
	
	Portanto, temos apenas um valor $\lambda = 1$ com multiplicidade 3.
	Procurando os autovetores:
	
	$$
	\begin{pmatrix}
		4 & 3 & -2 & 0\\
		0 & 0 &  0 & 0\\
		8 & 6 & -4 & 0
	\end{pmatrix}
	\begin{array}{c}
		\\ \\ L_3 = L_3 - 2L_1
	\end{array}
	\begin{pmatrix}
		4 & 3 & -2 & 0\\
		0 & 0 &  0 & 0\\
		0 & 0 &  0 & 0
	\end{pmatrix}
	4x+3y-2z = 0
	$$
	Isolando $x$ na equação encontrada, podemos encontrar
	a base do plano que determina o auto-esaço da matriz.
	$$
	x = -\frac{3y}{4} + \frac{z}{2}	
	$$
	$$
	(x, \, y, \, z) = y(-3, \, 4, \, 0) + z(1, \, 0, \, 2)	
	$$
	
	Desta base, podemos tirar os autovetores:
	
	$$
	\begin{array}{ccc}
		& \lambda = -1 & \\
		v_1 = (-3, \, 4, \, 0) & & v_2 = (1, \, 0, \, 2)
	\end{array}		
	$$
	
	
\end{enumerate}