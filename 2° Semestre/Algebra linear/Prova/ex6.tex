\item Determinar uma matriz P que diagonaliza A
e calcular $P^{-1}AP$. Calcule $A^{30}$ e $B^{101}$.

\begin{enumerate}
	%	LETRA A 	%
	\item
	$
	\begin{bmatrix}
		0 &  0 & 2 \\
		0 & -1 & 0 \\
		2 &  0 & 0
	\end{bmatrix}
	$
	\\ \\
	\solucao
	$$
	det 
	\begin{pmatrix}
		-\lambda &  0 & 2 \\
		0        & -1 - \lambda & 0 \\
		2 &  0 & - \lambda
	\end{pmatrix}
	= -\lambda
	\begin{vmatrix}
		-1 -\lambda & 0 \\
		0 & -\lambda
	\end{vmatrix}
	-0
	\begin{vmatrix}
		0 & 0 \\
		2 & -\lambda
	\end{vmatrix}
	+2
	\begin{vmatrix}
		0 & -1 -\lambda \\
		2 & 0
	\end{vmatrix}
	$$

	$$-\lambda^2(1+\lambda)+4(1+\lambda) = (1+\lambda)(4-\lambda^2) = 0$$
	
	$$
	\begin{array}{ccc}
		\lambda_1 = -2 & \lambda_2 = -1 & \lambda_3 = 2
	\end{array}
	$$
	
	Agora, basta encontrar os autovetores.
	
	$$
	\begin{pmatrix}
		2 & 0 & 2 & 0 \\
		0 & 1 & 0 & 0 \\
		2 & 0 & 2 & 0
	\end{pmatrix}		
	\begin{array}{c}
		x = -z \\
		y = 0 \\
	\end{array}		
	v_1 = (1, \, 0, \, -1)	
	$$
	
	$$
	\begin{pmatrix}
		1 & 0 & 2 & 0 \\
		0 & 0 & 0 & 0 \\
		2 & 0 & 1 & 0
	\end{pmatrix}		
	\begin{array}{c}
		\\ \\ L_3 = 2L_3 - L_1
	\end{array}	
	\begin{pmatrix}
		1 & 0 & 2 & 0 \\
		0 & 0 & 0 & 0 \\
		3 & 0 & 0 & 0
	\end{pmatrix}
	\begin{array}{c}
		z = 0 \\
		\\
		x = 0
	\end{array}	
	v_2 = (0, \, 1, \, 0)	
	$$
	
	$$
	\begin{pmatrix}
		-2 & 0 & 2 & 0 \\
		0 & -3 & 0 & 0 \\
		2 & 0 & -2 & 0
	\end{pmatrix}		
	\begin{array}{c}
		L_1 = \frac{L_1}{2} \\
		\\
		L_3 = L_1 + L_3
	\end{array}	
	\begin{pmatrix}
		-1 & 0 & 1 & 0 \\
		0 & -3 & 0 & 0 \\
		0 & 0 & 0 & 0
	\end{pmatrix}
	\begin{array}{c}
		x = z \\
		y = 0
	\end{array}	
	v_3 = (1, \, 0, \, 1)	
	$$
	
	A matriz $P$ é a matriz cujas colunas
	são os autovetores de $A$, ou seja, 
	$P = \begin{bmatrix}
		1 & 0 & 1 \\
		0 & 1 & 0 \\
		-1 & 0 & 1
	\end{bmatrix}
	$. Para encontrar $P^{-1}$ basta transformar
	$P$ em $I$ ao lado da matri $I$:
	
	$$
	\begin{pmatrix}
		1  & 0 & 1 & 1 & 0 & 0\\
		0  & 1 & 0 & 0 & 1 & 0\\
		-1 & 0 & 1 & 0 & 0 & 1
	\end{pmatrix}
	\begin{array}{c}
		L_1 = \frac{L_1-L_3}{2} \\ \\
		L_3 = \frac{L_1 + L_3}{2}
	\end{array}
	\begin{pmatrix}
		1 & 0 & 0 & \frac{1}{2} & 0 & -\frac{1}{2}\\
		0 & 1 & 0 & 0 & 1 & 0\\
		1 & 0 & 1 & \frac{1}{2} & 0 & \frac{1}{2}
	\end{pmatrix}
	$$
	
	Desta forma, $P^{-1} = 
	\begin{pmatrix}
		\frac{1}{2} & 0 & -\frac{1}{2}\\
		0 & 1 & 0\\
		\frac{1}{2} & 0 & \frac{1}{2}
	\end{pmatrix}$, com isso podemos calcular $P^{-1}AP$ e checar
	que esta matriz é a matriz cuja diagonal principal são os autovalores de $A$
	e o resto dos valores são todos nulos.
	
	$$
	P^{-1}AP = 
	\begin{pmatrix}
		\frac{1}{2} & 0 & -\frac{1}{2}\\
		0 & 1 & 0\\
		\frac{1}{2} & 0 & \frac{1}{2}
	\end{pmatrix}
	\begin{pmatrix}
		0 & 0 & 2\\
		0 & -1 & 0\\
		2 & 0 & 0
	\end{pmatrix}
	\begin{pmatrix}
		1 & 0 & 1\\
		0 & 1 & 0\\
		-1 & 0 & 1
	\end{pmatrix}
	=
	\begin{pmatrix}
		-1 & 0 & 1\\
		0 & -1 & 0\\
		1 & 0 & 1
	\end{pmatrix}
	\begin{pmatrix}
		1 & 0 & 1\\
		0 & 1 & 0\\
		-1 & 0 & 1
	\end{pmatrix}
	$$
	
	$$
	P^{-1}AP =
	\begin{pmatrix}
		-2 & 0 & 0\\
		0 & 1 & 0\\
		0 & 0 & 2
	\end{pmatrix}
	=
	\begin{pmatrix}
		\lambda_1 & 0 & 0\\
		0 & \lambda_2 & 0\\
		0 & 0 & \lambda_3
	\end{pmatrix}
	$$
	
	Finalmente, utilizarei a matri diagonalizada para
	calcular $A^{30}$. Isso pois, multiplicações com
	matrizes diagonalizadas são muito mais fáceis.
	Além disso, sendo $D$ a matriz $A$ diagonalizada, da
	relação $D = P^{-1}AP$, podemos encontrar $A = PDP{-1}$.
	Portanto, após encontrar $D^{30}$, encontrar $A^{30}$ é um processo fácil.
	
	$$
	D^2 = 
	\begin{pmatrix}
		-2 & 0 & 0\\
		0 & 1 & 0\\
		0 & 0 & 2
	\end{pmatrix}
	\begin{pmatrix}
		-2 & 0 & 0\\
		0 & 1 & 0\\
		0 & 0 & 2
	\end{pmatrix}
	=
	\begin{pmatrix}
		4 & 0 & 0\\
		0 & 1 & 0\\
		0 & 0 & 4
	\end{pmatrix}	
	$$
	
	$$
	D^3 = 
	\begin{pmatrix}
		4 & 0 & 0\\
		0 & 1 & 0\\
		0 & 0 & 4
	\end{pmatrix}
	\begin{pmatrix}
		-2 & 0 & 0\\
		0 & 1 & 0\\
		0 & 0 & 2
	\end{pmatrix}
	=
	\begin{pmatrix}
		-8 & 0 & 0\\
		0 & 1 & 0\\
		0 & 0 & 8
	\end{pmatrix}	
	$$
	Com isso, podemos perceber que elevar uma matriz diagonalizada
	a um número n é o mesmo que elevar os valores de sua diagonal
	principal a n, assim::
	
	$$
	D^{30} = 
	\begin{pmatrix}
		(-2)^{30} & 0 & 0\\
		0 & 1 & 0\\
		0 & 0 & 2^{30}
	\end{pmatrix}$$
	
	Por fim, basta encontrar $A^{30}$
	
	$$
	A^{30} = PD^{30}P^{-1}
	=
	\begin{pmatrix}
		1 & 0 & 1\\
		0 & 1 & 0 \\
		-1 & 0 & 1 
	\end{pmatrix}
	\begin{pmatrix}
		(-2)^{30} & 0 & 0\\
		0 & 1 & 0\\
		0 & 0 & 2^{30}
	\end{pmatrix}
	\begin{pmatrix}
		\frac{1}{2} & 0 & -\frac{1}{2}\\
		0 & 1 & 0\\
		\frac{1}{2} & 0 & \frac{1}{2}
	\end{pmatrix}
	$$
	
	$$
	A^{30} = 
	\begin{pmatrix}
		2^{30} & 0 & 2^{30} \\
		0 & 1 & 0 \\
		-2^{30} & 0 & 2^{30}
	\end{pmatrix}
	\begin{pmatrix}
		\frac{1}{2} & 0 & -\frac{1}{2}\\
		0 & 1 & 0\\
		\frac{1}{2} & 0 & \frac{1}{2}
	\end{pmatrix}
	=
	\begin{pmatrix}
		2^{30} & 0 & 0 \\
		0 & 1 & 0 \\
		0 & 0 & 2^{30}
	\end{pmatrix}
	$$
	
	Desta forma, temos como respostas:
	
	$$
	\begin{array}{ccc}
		P = 
		\begin{pmatrix}
			1 & 0 & 1 \\
			0 & 1 & 0 \\
			-1 & 0 & 1
		\end{pmatrix}
		&
		P^{-1}AP = 
		\begin{pmatrix}
			-2 & 0 & 0 \\
			0 & -1 & 0 \\
			0 & 0 &  2
		\end{pmatrix}
		&
		A^{30} = 
		\begin{pmatrix}
			1073741824 & 0 & 0 \\
			0 & 1 & 0 \\
			0 & 0 & 1073741824
		\end{pmatrix}
	\end{array}				
	$$
	
	%	LETRA B 	%
	\pagebreak
	\item
	$
	\begin{bmatrix}
		2 &  -2 & 1 \\
		-2 & 2 & 1 \\
		-1 &  1 & 5
	\end{bmatrix}
	$
	\\ \\
	\solucao
	
	$$
	det
	\begin{pmatrix}
		 2 - \lambda &  -2 & 1 \\
		-2 & 2 - \lambda & 1 \\
		-1 &  1 & 5 - \lambda
	\end{pmatrix}
	=
	(2 - \lambda)
	\begin{vmatrix}
		2 - \lambda & 1 \\
		1 & 5 - \lambda	
	\end{vmatrix}
	+ 2
	\begin{vmatrix}
		-2 & 1 \\
		-1 & 5 - \lambda
	\end{vmatrix}
	- 1
	\begin{vmatrix}
		-2 & 2 - \lambda \\
		-1 & 1
	\end{vmatrix}
	$$
	
	$$(2 - \lambda)(\lambda^2 - 7\lambda+9)+2(2\lambda-9)+\lambda = 0$$
	
	$$\lambda^3 - 9\lambda^2 - 18\lambda = \lambda(\lambda^2 -9\lambda-18) = 0$$
		
	$$
	\begin{array}{cc}
		\Delta = 81 - 72 = 9 & \lambda = \dfrac{9 \pm 3}{2}
	\end{array}	
	$$

	$$
	\begin{array}{ccc}
		\lambda_1 = 0 & \lambda_2 = 3 & \lambda = 6
	\end{array}
	$$
	
	Agora encontrando os autovetores:
	
	$$
	\begin{pmatrix}
		2 &  -2 & 1 & 0\\
		-2 & 2 & 1 & 0\\
		-1 &  1 & 5 & 0
	\end{pmatrix}	
	\begin{array}{c}
		\\ L_2 = L_2 + L_1 \\ L_3 = 2L_3 + L_1
	\end{array}
	\begin{pmatrix}
		2 &  -2 & 1 & 0\\
		0 & 0 & 0 & 0\\
		0 & 0 & 9 & 0
	\end{pmatrix}
	\begin{array}{cc}
		x = y\\ & v_1 = (1, \, 1, \, 0) \\ z = 0
	\end{array}
	$$
	
	$$
	\begin{pmatrix}
		-1 &  -2 & 1 & 0\\
		-2 & -1 & 1 & 0\\
		-1 &  1 & 2 & 0
	\end{pmatrix}	
	\begin{array}{c}
		\\ L_2 = L_2 + L_1 \\ L_3 = L_3 + 2L_1
	\end{array}
	\begin{pmatrix}
		-1 & -2 & 1 & 0\\
		-3 & -3 & 0 & 0\\
		-3 & -3 & 0 & 0
	\end{pmatrix}
	\begin{array}{c}
		-x-2y-z=0 \Rightarrow y = -z \\ x = -y \\ v_2 = (1, \, -1, \, 1)
	\end{array}
	$$

	$$
	\begin{pmatrix}
		-4 &  -2 & 1 & 0\\
		-2 & -4 & 1 & 0\\
		-1 &  1 & -1 & 0
	\end{pmatrix}	
	\begin{array}{c}
		L_1 = - L_3\\ L_2 = L_2 - 2L_3 \\ L_3 = L_1 -4 L_3
	\end{array}
	\begin{pmatrix}
		1 & -1 & 1 & 0\\
		0 & -6 & 3 & 0\\
		0 & -6 & 3 & 0
	\end{pmatrix}
	\begin{array}{c}
		x-y+z=0 \Rightarrow x = -y \\ z = 2y \\ v_3 = (1, \, -1, \, -2)
	\end{array}
	$$
	
	Encontrados os três autovetores, podemos escrever a matriz $P$,
	colocando cada um dos autovetores em uma coluna, assim 
	$P = 
	\begin{pmatrix}
		1 & 1 & 1 \\
		1 & -1 & -1 \\
		0 & 1 & -2
	\end{pmatrix}
	$. Com isso, podemos encontrar $P^{-1}$ e verificar que
	$P^{-1}BP$ é uma matriz cuja diagonal principal são os autovalores
	de $B$ e os outros elementos são nulos.
	
	$$
	\begin{pmatrix}
		1 & 1 & 1  & 1 & 0 & 0\\
		1 & -1 & -1 & 0 & 1 & 0\\
		0 & 1 & -2 & 0 & 0 & 1
	\end{pmatrix}
	\begin{array}{c}
		L_1 = \dfrac{L_1 + L_2}{2} \\
		L_2 = L_1 - L_2 \\
	\end{array}
	\begin{pmatrix}
		1 & 0 & 0  & \dfrac{1}{2} & \dfrac{1}{2} & 0\\
		0 & 2 & 2 & 1 & -1 & 0\\
		0 & 1 & -2 & 0 & 0 & 1
	\end{pmatrix}
	$$
	
	$$
	\begin{pmatrix}
		1 & 0 & 0  & \dfrac{1}{2} & \dfrac{1}{2} & 0\\
		0 & 2 & 2 & 1 & -1 & 0\\
		0 & 1 & -2 & 0 & 0 & 1
	\end{pmatrix}
	\begin{array}{c}
		L_2 = \dfrac{L_2 + L_3}{3} \\
		L_3 = \dfrac{L_2 - 2L_3}{6}
	\end{array}
	\begin{pmatrix}
		1 & 0 & 0  & \frac{1}{2} & \frac{1}{2} & 0\\
		0 & 1 & 0 & \frac{1}{3} & -\frac{1}{3} & \frac{1}{3}\\
		0 & 0 & 1 & \frac{1}{6} & -\frac{1}{6} & -\frac{1}{3}
	\end{pmatrix}
	$$
	
	Finalmente, encontrando $P^{-1}AP$:
	
	$$
	P^{-1}BP = 
	\begin{pmatrix}
		\frac{1}{2} & \frac{1}{2} & 0\\
		\frac{1}{3} & -\frac{1}{3} & \frac{1}{3}\\
		\frac{1}{6} & -\frac{1}{6} & -\frac{1}{3}
	\end{pmatrix}
	\begin{pmatrix}
		2 &  -2 & 1 \\
		-2 & 2 & 1 \\
		-1 &  1 & 5
	\end{pmatrix}
	\begin{pmatrix}
		1 & 1 & 1 \\
		1 & -1 & -1 \\
		0 & 1 & -2
	\end{pmatrix}
	$$
	
	$$
	P^{-1}BP = 
	\begin{pmatrix}
		\frac{1}{2} & \frac{1}{2} & 0\\
		\frac{1}{3} & -\frac{1}{3} & \frac{1}{3}\\
		\frac{1}{6} & -\frac{1}{6} & -\frac{1}{3}
	\end{pmatrix}
	\begin{pmatrix}
		0 & 3 & 6 \\
		0 & -3 & -6 \\
		0 & 3 & -12
	\end{pmatrix}
	=
	\begin{pmatrix}
		0 & 0 & 0 \\
		0 & 3 & 0 \\
		0 & 0 & 6
	\end{pmatrix}
	= 
	\begin{pmatrix}
		\lambda_1 & 0 & 0 \\
		0 & \lambda_2 & 0 \\
		0 & 0 & \lambda_3
	\end{pmatrix}
	$$
	
	O processo para encontrar $B^{101}$ será o mesmo que
	foi utilizado para encontrar $B^{30}$.
	
	$$
	D^{101} =
	\begin{pmatrix}
		0 & 0 & 0 \\
		0 & 3^{101} & 0 \\
		0 & 0 & 6^{101}
	\end{pmatrix}	
	$$
	
	$$
	B^{101} = PD^{101}P{-1} = 
	\begin{pmatrix}
		1 & 1 & 1 \\
		1 & -1 & -1 \\
		0 & 1 & -2
	\end{pmatrix}
	\begin{pmatrix}
		0 & 0 & 0 \\
		0 & 3^{101} & 0 \\
		0 & 0 & -6^{101}
	\end{pmatrix}
	\begin{pmatrix}
		\frac{1}{2} & \frac{1}{2} & 0\\
		\frac{1}{3} & -\frac{1}{3} & \frac{1}{3}\\
		\frac{1}{6} & -\frac{1}{6} & -\frac{1}{3}
	\end{pmatrix}
	$$
	
	$$
	B^{101}=
	\begin{pmatrix}
		0 &  3^{101} &  6 ^{101} \\
		0 & -3^{101} & -6^{101} \\
		0 &  3^{101}  & -2 \cdot 6^{-101}
	\end{pmatrix}
	\begin{pmatrix}
		\frac{1}{2} & \frac{1}{2} & 0\\
		\frac{1}{3} & -\frac{1}{3} & \frac{1}{3}\\
		\frac{1}{6} & -\frac{1}{6} & -\frac{1}{3}
	\end{pmatrix}
	$$
	
	$$
	B^{101} = 
	\begin{pmatrix}
		3^{100}(1+2^{100}) & -3^{100}(1+2^{100}) & 3^{100}(1-2^{101}) \\
		-3^{100}(1+2^{100}) & 3^{100}(1+2^{100}) & -3^{100}(1-2^{101}) \\
		3^{100}(1-2^{101}) & -3^{100}(1-2^{101}) & 3^{100}(1+2^{102}) \\
	\end{pmatrix}
	$$
	
	Assim, as respostas das perguntas são:
	
	$$
	P = 
	\begin{pmatrix}
		1 & 1 & 1 \\
		1 & -1 & -1 \\
		0 & 1 & -2
	\end{pmatrix}
	$$
	
	$$
	P^{-1}BP = 
	\begin{pmatrix}
		0 & 0 & 0 \\
		0 & 3 & 0 \\
		0 & 0 & 6
	\end{pmatrix}
	$$
	
	$$
	B^{101} =
	\begin{pmatrix}
		 6,5\cdot 10^{77} & - 6,5 \cdot 10^{77} & -1,3 \cdot 10^78 \\
		-6,5\cdot 10^{77} &   6,5 \cdot 10^{77} & 1,3 \cdot 10^78 \\
		-1,3 \cdot 10^{78} & - 1,3 \cdot 10^{78} & 2,6 \cdot 10^{78}
	\end{pmatrix}	
	$$
\end{enumerate}