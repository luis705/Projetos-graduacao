\item 

\begin{enumerate}
\item 
Podemos utilizar a fórmula:

$$
v = \frac{F}{B}
$$

Assim, colocando os valores:

$$
v = \frac{5 \cdot 10^{-6}}{10,5 \cdot 10^{-3}} = 4,76 \cdot 10^{-4} m/s
$$

\item
Primeiro encontrando a energia em Joules:

$$
E = 150 \cdot 10^{6} \cdot 1,6 \cdot 10^{-19} = 2,4 \cdot 10^{-11}
$$

Em seguida, utilizando a fórmula de energia cinética:

$$
E = \frac{mv^2}{2} \Rightarrow m = \frac{2E}{v^2}
$$

Substituindo os valores:

$$
m = \frac{4,8 \cdot 10^{-11}}{(0,476 \cdot 10^{-3})^2} = 2,11 \cdot 10^{-4} Kg
$$

\item
Utilizando a fórmula de raio:

$$
r = \frac{mv}{|q|B} \Rightarrow |q| = \frac{mv}{rB}
$$

Novamente, substituindo os valores:

$$
|q| = \frac{2,11 \cdot 10^{-4} \cdot 4,76 \cdot 10^{-4}}
{10,3 \cdot 10^{-6} \cdot 10,5 \cdot 10^{-3}}
= 0,92C
$$

\item
Positivo
\end{enumerate}