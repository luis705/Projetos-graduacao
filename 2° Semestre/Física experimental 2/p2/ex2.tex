\item
\begin{enumerate}
\item
Basta encontrar o campo elétrico de
cada arco e somá-los:

$$
\vec{B_R} = \vec{B_1} + \vec{B_2}
$$

Assim, precisamos apenas calcular $B_1$ e $B_2$.

$$
B_1 = \frac{\mu_0 \cdot 5 \cdot \varphi}{4 \pi \cdot 3 \cdot 10^{-2}}\
= 55 \pi \cdot 10^{-5}
$$

Detalhe: o angulo foi fornecido em gruas, por isso
foi convertido para radianos:

$$
\varphi = \frac{330 \pi}{180} = \frac{33 \pi}{18}
$$

Calculando $B_2$:

$$
B_2 = \frac{\mu_0 \cdot 5 \cdot \pi}{4 \pi \cdot 10^{-2} \cdot 9 \cdot 6}\
= \frac{5 \pi \cdot 10^{-7}}{54}
$$

Assim, somando os dois campos:

$$
B_R = \frac{2975 \pi \cdot 10^{-7}}{54}
$$

\item Perpendicular à folha, saindo da página.
\end{enumerate}
